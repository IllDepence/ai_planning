\documentclass[11pt,a4paper]{article}
\usepackage[a4paper, margin=1.3in]{geometry}
\usepackage{mathtools}
\usepackage{fancyhdr}
\usepackage{mathrsfs}

\newcommand{\sheetNr}{8}

\pagestyle{fancy}
\fancyhf{}
\lhead{AI Planning}
\rhead{Exercise Sheet \sheetNr}
\lfoot{Axel Perschmann, Tarek Saier, \today}
\rfoot{Page \thepage\ of \pageref{lastpage}}
\renewcommand{\headrulewidth}{0.3pt}
\renewcommand{\footrulewidth}{0.3pt}
\setlength\parindent{0pt}
\newcommand{\h}[0]{\text{--}}

\begin{document}
\begin{center}
\Huge{\textbf{AI Planning}}\\
\LARGE{\textbf{Exercise Sheet \sheetNr}}
\end{center}
\vspace{2cm}
\begin{tabular}{ll}
Date: & \today\\
Students: & Axel Perschmann, Tarek Saier
\end{tabular}

\section*{Exercise 8.1}
General idea: heuristic values are completely dependent on the chosen projection $\pi_P$.\\
Example:\\
The heuristic value for the abstraction induced by $\pi_{\text{\{truck A, truck B}\}}$ of $I$ is 0:\\
$h^{\text{\{truck A, truck B\}}}(LRR)=0$, as opposed to:\\
$h^{\text{\{package, truck A\}}}(LRR)=2$\\
$h^{\text{\{package, truck B\}}}(LRR)=2$\\
\\
The above illustrates that it is important to choose the pattern wisely. Prune variables that are less significant, keep those that are most relevant for the goal formula.\\

\section*{Exercise 8.2}
Definitions/Notes:
\begin{itemize}
\item (1) $\Pi=\langle V,I,O,\gamma\rangle$ (in FDR) is SAS$^+$ iff\\
\hphantom{tabtab}- $\forall o\in O$ have no conditional effects\\
\hphantom{tabtab}- $\forall \chi$ of $o\in O$ and $\gamma$ are conjunctions of atoms
\item (2) $\mathscr{T}(\Pi)=\langle S,L,T,s_0,S_\star\rangle$ is the induced transition system\\
of $\Pi=\langle V,I,O,\gamma\rangle$ where\\
\hphantom{tabtab}- $S$ is the set of states over $V$\\
\hphantom{tabtab}- $L=O$\\
\hphantom{tabtab}- $T=\{\langle s,o,t\rangle\in S\times L\times S|app_o(s)=t\}$\\
\hphantom{tabtab}- $s_0=I$\\
\hphantom{tabtab}- $S_\star=\{s\in S|s\models\gamma\}$
\item (3) For $\mathscr{T}=\langle S,L,T,s_0,S_\star\rangle$ and $\alpha:S\to S'$ (a surjective function),\\
$\mathscr{T}^\alpha=\langle S',L,T',s'_0,S'_\star\rangle$ is the abstraction of $\mathscr{T}$ induced by $\alpha$ where\\
\hphantom{tabtab}- $S'=\{\alpha(s)|s\in S\}$\\
\hphantom{tabtab}- $T=\{\langle \alpha(s),\ell,\alpha(t)\rangle|\langle s,\ell,t\rangle\in T\}$\\
\hphantom{tabtab}- $s'_0=\alpha(s_0)$\\
\hphantom{tabtab}- $S'_\star=\{\alpha(s)|s\in S_\star\}$
\item (4) $P\subseteq V$ is a pattern, $\Pi|_P$ denotes $\Pi$ restricted to the variables in $P$
\item (5) $\pi_P$ is the projection $S\to S'$ for $P$, $\pi_P(s):=s|_P$, $\mathscr{T}^{\pi_P}$ denotes the transition\\
system induced by $\pi_P$
\item (6) two transition systems $\mathscr{T}$ and $\mathscr{T}'$ are graph-equivalent ($\mathscr{T}\stackrel{G}{\sim}\mathscr{T}'$) if there exists a bijective function $\phi:S\to S'$ such that\\
\hphantom{tabtab}- $\phi(s_0)=s'_0$\\
\hphantom{tabtab}- $s\in S_\star$ iff $\phi(s)\in S'_\star$\\
\hphantom{tabtab}- $\langle s,\ell,t\rangle\in T$ for \emph{some} $\ell\in L$ iff $\langle\phi(s),\ell',\phi(t)\rangle\in T'$ for \emph{some} $\ell'\in L'$\\
\end{itemize}
\textbf{(a)} to show: $\mathscr{T}(\Pi|_P)\stackrel{G}{\sim}\mathscr{T}(\Pi)^{\pi_P}$ if $\Pi$\\
\hphantom{tabtab}- is an SAS$^+$ planning task\\
\hphantom{tabtab}- is not trivially unsolvable\\
\hphantom{tabtab}- has no trivially inapplicable operators\\
\\
In terms of the definitions/notes above: (6) holds between:\\
\hphantom{tabtab}- (4), then (2)\\
\hphantom{tabtab}and\\
\hphantom{tabtab}- (2), then (5) --- the latter being a special case of (3)\\
\\
$\mathscr{T}(\Pi|_P)$:\\
The difference to $\mathscr{T}(\Pi)$ in this case is $V$ of $\Pi|_P$, which affects $S$ directly and $T$ and $S_\star$ indirectly. $S$ will not include states over $V\setminus P$. $T$ and $S_\star$ will be restricted accordingly.\\
\\
$\mathscr{T}(\Pi)^{\pi_P}$:\\
The difference to $\mathscr{T}(\Pi)$ in this case is that after getting $\mathscr{T}$ the latter is abstracted by means of $\pi_P$. This means for all states $s'\in S'$ of $\mathscr{T}^{\pi_P}$, $s\in S$ of $\mathscr{T}$: $s'=s|_P$. In other words, information about variables $v\notin P$ will be discarded in all $s'$. Again, $T$ and $S_\star$ will be restricted accordingly.\\
\\
In both cases all information about variables not in $P$ is discarded. Performing this step by means of (4) before (2) or by means of (5) after (2) does not violate (6).\\
\\
\textbf{(b)} Replacing effect conditions in $o\in O$ with $\top$ causes problems with the third (see (6) above) condition of graph-equivalence that deals with transitions.

\label{lastpage}
\end{document}

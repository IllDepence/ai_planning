\documentclass[11pt,a4paper]{article}
\usepackage[a4paper, margin=1.3in]{geometry}
\usepackage{mathtools}
\usepackage{graphicx}
\usepackage{fancyhdr}
\pagestyle{fancy}
\fancyhf{}
\lhead{AI Planning}
\rhead{Exercise Sheet 5}
\lfoot{Axel Perschmann, Tarek Saier, 27.11.2014}
\rfoot{Page \thepage\ of n}
\renewcommand{\headrulewidth}{0.3pt}
\renewcommand{\footrulewidth}{0.3pt}
\setlength\parindent{0pt}

\newcommand{\h}[0]{\text{--}}

\begin{document}
\begin{center}
\Huge{\textbf{AI Planning}}\\
\LARGE{\textbf{Exercise Sheet 5}}
\end{center}
\vspace{2cm}
\begin{tabular}{ll}
Date: & 27.11.2014\\
Students: & Axel Perschmann, Tarek Saier
\end{tabular}

\section*{Exercise 5.1}
Proof by induction over the structure of $\chi$.

Base case $\chi = \top$: then $s' \models \top$.\\
Base case $\chi = \bot$: then $s \not\models \bot$.\\

Base case $\chi = a \in A$: assume $s \models a$ and $on(s) \subseteq on(s')$. \\
Wich $a \in on(s)$ we get $a \in on(s')$, hence $s' \models a$.\\

Inductive case $\chi = \chi_1 \land \chi_2$

\begin{tabular}{l l}
$s \models \chi$ & $\iff  s \models \chi_1 \land \chi_2$ \\
 & $\iff  s \models \chi_1$ and $s \models \chi_2$\\
 & $\implies s' \models \chi_1$ and $s' \models \chi_2$\\
 & $\iff s' \models \chi_1$ and $\chi_2$ \\
 & $\iff s' \models \chi$ 
\end{tabular}
 
Inductive case $\chi = \chi_1 \lor \chi_2$ (Analogous to previous case)

\begin{tabular}{l l}
$s \models \chi$ & $\iff  s \models \chi_1 \lor \chi_2$ \\
 & $\iff  s \models \chi_1$ or $s \models \chi_2$\\
 & $\implies s' \models \chi_1$ or $s' \models \chi_2$\\
 & $\iff s' \models \chi_1$ or $\chi_2$ \\
 & $\iff s' \models \chi$ 
\end{tabular}
                




\section*{Exercise 5.2}
(a) $\Pi^+ = \langle A,I,O,\gamma \rangle$ with $A$, $I$, $\gamma$ unchanged and\\
\begin{tabular}{rcl}
$O$ & $=$ & $\{eatCake^+,bakeCake^+\}$\\
$eatCake^+$ & $=$ & $\langle haveCake, \top \land haveNoCake \land eatenCake\rangle$\\
$bakeCake^+$ & $=$ & $\langle haveNoCake, haveCake \land \top \rangle$\\
\end{tabular}\\
\\
\\
(b) $\pi = bakeCake, eatCake$\\
$\pi$ in $\Pi$ results in $\{haveCake \mapsto 0, eatenCake \mapsto 1, haveNoCake \mapsto 1\}$\\
$\pi^+$ in $\Pi^+$ results in $\{haveCake \mapsto 1, eatenCake \mapsto 1, haveNoCake \mapsto 1\}$\\

\end{document}

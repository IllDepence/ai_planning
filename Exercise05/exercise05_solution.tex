\documentclass[11pt,a4paper]{article}
\usepackage[a4paper, margin=1.3in]{geometry}
\usepackage{mathtools}
\usepackage{graphicx}
\usepackage{fancyhdr}
\pagestyle{fancy}
\fancyhf{}
\lhead{AI Planning}
\rhead{Exercise Sheet 5}
\lfoot{Axel Perschmann, Tarek Saier, 27.11.2014}
\rfoot{Page \thepage\ of 2}
\renewcommand{\headrulewidth}{0.3pt}
\renewcommand{\footrulewidth}{0.3pt}
\setlength\parindent{0pt}

\newcommand{\h}[0]{\text{--}}

\begin{document}
\begin{center}
\Huge{\textbf{AI Planning}}\\
\LARGE{\textbf{Exercise Sheet 5}}
\end{center}
\vspace{2cm}
\begin{tabular}{ll}
Date: & 27.11.2014\\
Students: & Axel Perschmann, Tarek Saier
\end{tabular}

\section*{Exercise 5.1}
As given in the lecture slides:\\
\\
Proof by induction over the structure of $\chi$.

Base case $\chi = \top$: then $s' \models \top$.\\
Base case $\chi = \bot$: then $s \not\models \bot$.\\

Base case $\chi = a \in A$: assume $s \models a$ and $on(s) \subseteq on(s')$. \\
Wich $a \in on(s)$ we get $a \in on(s')$, hence $s' \models a$.\\

Inductive case $\chi = \chi_1 \land \chi_2$

\begin{tabular}{l l}
$s \models \chi$ & $\iff  s \models \chi_1 \land \chi_2$ \\
 & $\iff  s \models \chi_1$ and $s \models \chi_2$\\
 & $\implies s' \models \chi_1$ and $s' \models \chi_2$\\
 & $\iff s' \models \chi_1$ and $\chi_2$ \\
 & $\iff s' \models \chi$ 
\end{tabular}
 
Inductive case $\chi = \chi_1 \lor \chi_2$ (Analogous to previous case)

\begin{tabular}{l l}
$s \models \chi$ & $\iff  s \models \chi_1 \lor \chi_2$ \\
 & $\iff  s \models \chi_1$ or $s \models \chi_2$\\
 & $\implies s' \models \chi_1$ or $s' \models \chi_2$\\
 & $\iff s' \models \chi_1$ or $\chi_2$ \\
 & $\iff s' \models \chi$ 
\end{tabular}

\section*{Exercise 5.2}
(a) $\Pi^+ = \langle A,I,O,\gamma \rangle$ with $A$, $I$, $\gamma$ unchanged and\\
\begin{tabular}{rcl}
$O$ & $=$ & $\{eatCake^+,bakeCake^+\}$\\
$eatCake^+$ & $=$ & $\langle haveCake, \top \land haveNoCake \land eatenCake\rangle$\\
$bakeCake^+$ & $=$ & $\langle haveNoCake, haveCake \land \top \rangle$\\
\end{tabular}\\
\\
\\
(b) $\pi = bakeCake, eatCake$\\
$\pi$ in $\Pi$ results in $\{haveCake \mapsto 0, eatenCake \mapsto 1, haveNoCake \mapsto 1\}$\\
$\pi^+$ in $\Pi^+$ results in $\{haveCake \mapsto 1, eatenCake \mapsto 1, haveNoCake \mapsto 1\}$\\

\section*{Exercise 5.3}
Considerations:\\
$h^{Manhattan}(s)$ sums up distances of tiles to their target position.\\
$h^+(s)$ considers the relaxed planning task with relaxed operators. This means when a tile is moved it moves to the new position but also ''stays'' at the old ($at()$ for the old position is not negated). $\chi$ of each $o^+$ should be the same in $\Pi$ and $\Pi^+$ though.\\
\\
\textbf{(a)}\\
$h^{Manhattan}(s)$ is equal to $h^+(s)$ without the restictions given by $\chi$ for each $o^+$. In other words: for $h^{Manhattan}(s)$ you can any tile freely, for $h^+(s)$ you can move tiles freely along the paths where $empty(p_{i,j})$ was made true at one point or was true for $s$ in the first place. Therefore $h^{Manhattan}(s)$ can never yield a larger number than $h^+(s)$.\\
\\
\textbf{(b)}\\
$s = \begin{matrix}
t_0 & t_1 & t_2 & t_3 \\
t_4 & t_5 & t_6 & t_7 \\
t_8 & t_9 & t_{10} & t_{11} \\
t_{14} & t_{13} & t_{12} & - \\
\end{matrix}$\\
$h^{Manhattan}(s)=(|0-2|+|3-3|)+(|2-0|+|3-3|)=4$\\
$h^+(s)=7$ with plan:\\ % not super sure on that one
$move(t_{12},p_{2,3},p_{3,3})$\\
$move(t_{13},p_{1,3},p_{2,3})$\\
$move(t_{14},p_{0,3},p_{1,3})$\\
$move(t_{14},p_{1,3},p_{2,3})$\\
$move(t_{12},p_{3,3},p_{2,3})$\\
$move(t_{12},p_{2,3},p_{1,3})$\\
$move(t_{12},p_{1,3},p_{0,3})$

\end{document}

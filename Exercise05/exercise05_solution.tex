\documentclass[11pt,a4paper]{article}
\usepackage[a4paper, margin=1.3in]{geometry}
\usepackage{mathtools}
\usepackage{graphicx}
\usepackage{fancyhdr}
\pagestyle{fancy}
\fancyhf{}
\lhead{AI Planning}
\rhead{Exercise Sheet 5}
\lfoot{Axel Perschmann, Tarek Saier, dd.11.2014}
\rfoot{Page \thepage\ of n}
\renewcommand{\headrulewidth}{0.3pt}
\renewcommand{\footrulewidth}{0.3pt}
\setlength\parindent{0pt}

\newcommand{\h}[0]{\text{--}}

\begin{document}
\begin{center}
\Huge{\textbf{AI Planning}}\\
\LARGE{\textbf{Exercise Sheet 5}}
\end{center}
\vspace{2cm}
\begin{tabular}{ll}
Date: & dd.11.2014\\
Students: & Axel Perschmann, Tarek Saier
\end{tabular}

\section*{Exercise 5.1}
foo

\section*{Exercise 5.2}
(a) $\Pi^+ = \langle A,I,O,\gamma \rangle$ with $A$, $I$, $\gamma$ unchanged and\\
\begin{tabular}{rcl}
$O$ & $=$ & $\{eatCake^+,bakeCake^+\}$\\
$eatCake^+$ & $=$ & $\langle haveCake, \top \land haveNoCake \land eatenCake\rangle$\\
$bakeCake^+$ & $=$ & $\langle haveNoCake, haveCake \land \top \rangle$\\
\end{tabular}\\
\\
\\
(b) $\pi = bakeCake, eatCake$\\
$\pi$ in $\Pi$ results in $\{haveCake \mapsto 0, eatenCake \mapsto 1, haveNoCake \mapsto 1\}$\\
$\pi^+$ in $\Pi^+$ results in $\{haveCake \mapsto 1, eatenCake \mapsto 1, haveNoCake \mapsto 1\}$\\

\end{document}

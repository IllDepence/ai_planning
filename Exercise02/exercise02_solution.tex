\documentclass[11pt,a4paper]{article}
\usepackage[a4paper, margin=1.3in]{geometry}
\usepackage{mathtools}
\usepackage{listings}
\usepackage{fancyhdr}
\usepackage{xcolor}
\pagestyle{fancy}
\fancyhf{}
\lhead{AI Planning}
\rhead{Exercise Sheet 2}
\lfoot{Axel Perschmann, Tarek Saier, 06.11.2014}
\rfoot{Page \thepage\ of 3}
\renewcommand{\headrulewidth}{0.3pt}
\renewcommand{\footrulewidth}{0.3pt}
\setlength\parindent{0pt}

\begin{document}
\begin{center}
\Huge{\textbf{AI Planning}}\\
\LARGE{\textbf{Exercise Sheet 2}}
\end{center}
\vspace{2cm}
\begin{tabular}{ll}
Date: & 06.11.2014\\
Students: & Axel Perschmann, Tarek Saier
\end{tabular}

\section*{Exercise 2.1}
\subsection*{(a) Transform the operator}

original \newline
$ \langle
    \neg e \lor f ,
  (a \triangleright ( b \triangleright c))   \land
  (\neg d \triangleright c)  \land
  ( \neg (\neg c \land \neg a) \triangleright (d \land \neg d) )   \land
  (d \triangleright \neg e)  
  \rangle
$

(7) \newline
$ \langle
    \neg e \lor f ,
  ((a \land b) \triangleright c)   \land
  (\neg d \triangleright c)  \land
  ( \neg (\neg c \land \neg a) \triangleright (d \land \neg d) )   \land
  (d \triangleright \neg e)  
  \rangle
$

(9) \newline
$ \langle
    \neg e \lor f ,
  (((a \land b) \lor \neg d) \triangleright c)  \land
  ( \neg (\neg c \land \neg a) \triangleright (d \land \neg d) )   \land
  (d \triangleright \neg e)  
  \rangle
$

(8) \newline
$ \langle
    \neg e \lor f ,
  (((a \land b) \lor \neg d) \triangleright c)  \land
  ( \neg (\neg c \land \neg a) \triangleright d )   \land
  ( \neg (\neg c \land \neg a) \triangleright \neg e )   \land
  (d \triangleright \neg e)  
  \rangle
$

(9) \newline
$ \langle
    \neg e \lor f ,
  (((a \land b) \lor \neg d) \triangleright c)  \land
  ( \neg (\neg c \land \neg a) \triangleright d )   \land
  (( \neg (\neg c \land \neg a) \lor d) \triangleright \neg e)  
  \rangle
$

2x deMorgan \newline
$ \langle
    \neg e \lor f ,
  (((a \land b) \lor \neg d) \triangleright c)  \land
  (( c \lor a) \triangleright d )   \land
  (( c \lor a \lor d) \triangleright \neg e)
  \rangle  
$

\subsection*{(b) Transform the ENF operator}
original \newline
$ \langle
    \neg e \lor f ,
  (((a \land b) \lor \neg d) \triangleright c)   \land
  ((c \lor a) \triangleright d)  \land
  (( c \lor a \lor d) \triangleright \neg e)
  \rangle
$

identify negative literals in coditions \newline
$ \langle
    \textcolor{red}{\neg e} \lor f ,
  (((a \land b) \lor \textcolor{red}{\neg d}) \triangleright c)   \land
  ((c \lor a) \triangleright d)  \land
  (( c \lor a \lor d) \triangleright \neg e)
  \rangle
$

introduce $\hat{d}$=$\neg d$ and $\hat{e}$=$\neg e$ and add it for effects \newline
$ \langle
    \neg e \lor f ,
  (((a \land b) \lor \neg d) \triangleright c)   \land
  ((c \lor a) \triangleright \textcolor{red}{d \land \neg \hat{d}})  \land
  (( c \lor a \lor d) \triangleright \textcolor{red}{\neg e \land \hat{e}})
  \rangle
$

replace negative $d$ and $e$ with $\hat{d}$ and $\hat{e}$ in conditions \newline
$ \langle
    \textcolor{red}{\hat{e}} \lor f ,
  (((a \land b) \lor \textcolor{red}{\hat{d}}) \triangleright c)   \land
  ((c \lor a) \triangleright d \land \neg \hat{d})  \land
  (( c \lor a \lor d) \triangleright \neg e \land \hat{e})
  \rangle
$

(8) \newline
$ \langle
    \hat{e} \lor f ,
  (((a \land b) \lor \hat{d}) \triangleright c)   \land
  ((c \lor a) \triangleright d)  \land
  ((c \lor a) \triangleright \neg \hat{d})  \land
  (( c \lor a \lor d) \triangleright \neg e) \land
  (( c \lor a \lor d) \triangleright \hat{e})
  \rangle
$
\newpage

\section*{Exercise 2.2}
\textbf{(a-c)}
\begin{lstlisting}[frame=single,basicstyle=\ttfamily\footnotesize,caption={set cover problem as a PDDL domain}]
(define (domain set-cover)
    (:requirements :adl)
    (:types set elem)
    (:predicates (contains ?s - set ?e - elem)
        (selected ?s - set)
        (covered ?e - elem))
    (:action select-set
        :parameters(?s - set)
        :precondition (not (selected ?s))
        :effect (and (selected ?s)
                     (forall (?e - elem)
                        (when (contains ?s ?e)
                            (covered ?e))))))
\end{lstlisting}
\begin{lstlisting}[frame=single,basicstyle=\ttfamily\footnotesize,caption={set cover instance as a PDDL problem}]
(define (problem set-cover-i-1)
    (:domain set-cover)
    (:objects e1 e2 e3 e4 - elem s1 s2 s3 s4 s5 - set)
    (:init (contains s1 e1)
        (contains s2 e2)
        (contains s2 e3)
        (contains s3 e4)
        (contains s4 e1)
        (contains s4 e2)
        (contains s5 e3)
        (contains s5 e4))
    (:goal (and (covered e1)
                (covered e2)
                (covered e3)
                (covered e4))))
\end{lstlisting}
\begin{lstlisting}[frame=single,basicstyle=\ttfamily\footnotesize,caption={solving set cover instance with fast-downward}]
$ ./fast-downward.py scprob.pddl --search "astar(blind())"
[...]
select-set s4 (1)
select-set s5 (1)
Plan length: 2 step(s).
Plan cost: 2
Initial state h value: 1.
Expanded 6 state(s).
Reopened 0 state(s).
Evaluated 16 state(s).
Evaluations: 16
Generated 21 state(s).
Dead ends: 0 state(s).
Expanded until last jump: 1 state(s).
Reopened until last jump: 0 state(s).
Evaluated until last jump: 6 state(s).
Generated until last jump: 5 state(s).
Number of registered states: 16
Search time: 0s
Total time: 0s
Solution found.
Peak memory: 2924 KB
$ cat sas_plan
(select-set s4)
(select-set s5)
; cost = 2 (unit cost)
\end{lstlisting}
\textbf{(d)}\\
- Optimal planning: every possible plan is considered. It is guaranteed that the optimal plan (lowest cost) can be identified.\\
- Satisficing planning: it is sufficient to identify a plan that leads to the goal.\\
- Arbitrary set covers: as long as every element is covered we're satisfied.\\
- Cardinality minimal set covers: the goal is to pick $\mathcal{C} \subseteq \mathcal{S}$ such that every element is covered and $|\mathcal{C}|$ is minimal.\\
Since our formalization does only define complete coverage of elements as the goal, cardinal minimality can only be achieved by applying optimal planning. Since fast-downward increments the cost of a plan for each applied operation and the only operator defined is selecting an element of $\mathcal{S}$, cost of a plan equals to $|\mathcal{C}|$. Finding the optimal plan therefore means finding the minimal $\mathcal{C} \subseteq \mathcal{S}$ such that every element is covered.\\
In our formalization plan existence is equivalent to the existence of a set cover. If we were to include an element $e_5 \in \mathcal{U}$ such that $\forall S_i \in \mathcal{S} : e_5 \notin S_i$ there would be no possible set cover and therefore no existing plan that solves the problem.

\end{document}

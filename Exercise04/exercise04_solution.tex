\documentclass[11pt,a4paper]{article}
\usepackage[a4paper, margin=1.3in]{geometry}
\usepackage{mathtools}
\usepackage{graphicx}
\usepackage{fancyhdr}
\pagestyle{fancy}
\fancyhf{}
\lhead{AI Planning}
\rhead{Exercise Sheet 4}
\lfoot{Axel Perschmann, Tarek Saier, dd.11.2014}
\rfoot{Page \thepage\ of n}
\renewcommand{\headrulewidth}{0.3pt}
\renewcommand{\footrulewidth}{0.3pt}
\setlength\parindent{0pt}

\begin{document}
\begin{center}
\Huge{\textbf{AI Planning}}\\
\LARGE{\textbf{Exercise Sheet 4}}
\end{center}
\vspace{2cm}
\begin{tabular}{ll}
Date: & dd.11.2014\\
Students: & Axel Perschmann, Tarek Saier
\end{tabular}

\section*{Exercise 4.1}
For easy readability let the tiles be referred to as $b_1$, $b_2$, $w_1$ and $w_2$ and the empty cell be referred to as $e$. Furthermore, let the actions move and jump be denoted as $m_c(t)$ and $j_c(t)$ respectively where $c$ is the destination cell$\in \{1,2,3,4,5\}$ and $t$ is the tile that is being relocated.\\
As an example, the initial state is:\\
$b_1,b_2,w_1,w_2,e$\\
If we then apply $j_5(b_2)$ we reach:\\
$b_1,e,w_1,w_2,b_2$\\
\\
\textbf{(a)} Let $\lceil o\rceil$ be the search node $\sigma$ reached by applying the operation $o\in \{m_c(t),j_c(t)\}$.\\
$f(\lceil m_5(w_2)\rceil)=1+4=5$\\
$f(\lceil j_5(w_1)\rceil)=1+4=5$\\
$f(\lceil j_5(b_2)\rceil)=2+2=4$\\
Apply $j_5(b_2)$ which results in $\sigma_1$:\\
$b_1,e,w_1,w_2,b_2$\\
$f(\lceil m_2(b_1)\rceil)=3+2=5$\\
$f(\lceil m_2(w_1)\rceil)=3+2=5$\\
$f(\lceil j_2(w_2)\rceil)=3+2=5$\\
$\lceil j_2(b_2)\rceil=I\in closed$\\
Apply $m_2(b_1)$ which results in $\sigma_2$:\\
$e,b_1,w_1,w_2,b_2$\\
Apply $m_2(w_1)$ which results in $\sigma_3$:\\
$b_1,w_1,e,w_2,b_2$\\
Apply $j_2(w_2)$ which results in $\sigma_4$:\\
$b_1,w_2,w_1,e,b_2$\\
Expanding on $\sigma_2$:\\
$\lceil m_1(b_1)\rceil=\sigma_1\in closed$\\
$f(\lceil j_1(w_1)\rceil)=4+1=5$\\
$f(\lceil j_1(w_2)\rceil)=5+1=6$\\
Expanding on $\sigma_3$:\\
$f(\lceil j_3(b_1)\rceil)=4+1=5$\\
$f(\lceil m_3(w_1)\rceil)=4+2=6$\\
$f(\lceil m_3(w_2)\rceil)=4+2=6$\\
$f(\lceil j_3(b_2)\rceil)=4+3=7$\\
\\
Expanding on $\sigma_4$:\\
$f(\lceil j_4(b_1)\rceil)=5+0=5$\\
\\
Since $h$ is goal aware and the minimum cost of an operator is 1 we're done at this point. There may be other solutions but none with a cost of less than 5. The resulting plan is: $j_5(b_2),j_2(w_2),j_4(b_1)$ with a total cost of 5 a final state:\\
$e,w_2,w_1,b_1,b_2$\\
\\
\textbf{(b)}

\section*{Exercise 4.2}
bar

\end{document}

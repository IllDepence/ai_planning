\documentclass[11pt,a4paper]{article}
\usepackage[a4paper, margin=1.3in]{geometry}
\usepackage{mathtools}
\usepackage{graphicx}
\usepackage{fancyhdr}
\pagestyle{fancy}
\fancyhf{}
\lhead{AI Planning}
\rhead{Exercise Sheet 4}
\lfoot{Axel Perschmann, Tarek Saier, dd.11.2014}
\rfoot{Page \thepage\ of n}
\renewcommand{\headrulewidth}{0.3pt}
\renewcommand{\footrulewidth}{0.3pt}
\setlength\parindent{0pt}

\begin{document}
\begin{center}
\Huge{\textbf{AI Planning}}\\
\LARGE{\textbf{Exercise Sheet 4}}
\end{center}
\vspace{2cm}
\begin{tabular}{ll}
Date: & dd.11.2014\\
Students: & Axel Perschmann, Tarek Saier
\end{tabular}

\section*{Exercise 4.1}
For easy readability let the tiles be referred to as $b_1$, $b_2$, $w_1$ and $w_2$ and the empty cell be referred to as $e$. Furthermore, let the actions move and jump be denoted as $m_c(t)$ and $j_c(t)$ respectively where $c$ is the destination cell$\in \{1,2,3,4,5\}$ and $t$ is the tile that is being relocated.\\
As an example, the initial state is:\\
$b_1,b_2,w_1,w_2,e$\\
If we then apply $j_5(b_2)$ we reach:\\
$b_1,e,w_1,w_2,b_2$\\
\\
\textbf{(a)} Let $\lceil a\rceil$ be the search node $\sigma$ reached by applying the action $a\in \{m_c(t),j_c(t)\}$.\\
$f(\lceil m_5(w_2)\rceil)=1+4=5$\\
$f(\lceil j_5(w_1)\rceil)=1+4=5$\\
$f(\lceil j_5(b_2)\rceil)=2+2=4$\\
Apply $j_5(b_2)$ which results in:\\
$b_1,e,w_1,w_2,b_2$\\
% taking a break
\\
\textbf{(b)}

\section*{Exercise 4.2}
bar

\end{document}

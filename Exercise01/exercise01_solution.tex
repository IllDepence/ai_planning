\documentclass[11pt,a4paper]{article}
\usepackage{mathtools}
\setlength\parindent{0pt}

\begin{document}
\begin{center}
\Huge{\textbf{AI Planning}}\\
\LARGE{\textbf{Exercise Sheet 1}}
\end{center}
\vspace{2cm}
\begin{tabular}{ll}
Date: & 30.10.2014\\
Students: & Axel Perschmann, Tarek Saier
\end{tabular}

\section*{Exercise 1.1}
\begin{tabular}{ l | c c c }
     & districts & landing platforms & boys \\
  \hline
  King's Landing &  8 & 3 & 4\\
  Winterfell &  2 & 1 & 1\\
  Lannisport &  6 & 1 & 3\\
  Meereen & 12 & 1 & 6\\
  Volantis & 12 & 1 & 6\\
  \hline
  \#  & 40 & 7 & 20
\end{tabular}\\\\

\begin{tabular}{ll}
Number of states for errand boys & = $8^4 * 2^1 * 6^3 * 12^6 * 12^6$ \\
Number of states for dragons & = $7^3$ \\
Number of states for 30 packages &= $40^{30}$
\end{tabular}\\\\

The number of possible different states (size of the state space):\\
$state space size = (8^4 * 2 * 6^3 * 12^6 * 12^6)  *  7^3  * (40^{30})\\
state space size= 6.239 * 10^{69}$\\

Traverse time needed to visit all $6.239 * 10^{69}$ states:\\
$t = 6.239 * 10^{69} * 10^{-6}s\\
t = 6.239 * 10^{63}s$

\section*{Exercise 1.2}

\begin{enumerate}
\item How is a relaxed plan --- ''remembering'' old values and thus not actually representative of a real solution --- used to guide the search for an actual plan?
\item ?
\end{enumerate}


\end{document}

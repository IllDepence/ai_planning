\documentclass[11pt,a4paper]{article}
\usepackage[a4paper, margin=1.3in]{geometry}
\usepackage{mathtools}
\usepackage{fancyhdr}
\usepackage{mathrsfs}

\newcommand{\sheetNr}{7}

\pagestyle{fancy}
\fancyhf{}
\lhead{AI Planning}
\rhead{Exercise Sheet \sheetNr}
\lfoot{Axel Perschmann, Tarek Saier, \today}
\rfoot{Page \thepage\ of \pageref{lastpage}}
\renewcommand{\headrulewidth}{0.3pt}
\renewcommand{\footrulewidth}{0.3pt}
\setlength\parindent{0pt}
\newcommand{\h}[0]{\text{--}}

\begin{document}
\begin{center}
\Huge{\textbf{AI Planning}}\\
\LARGE{\textbf{Exercise Sheet \sheetNr}}
\end{center}
\vspace{2cm}
\begin{tabular}{ll}
Date: & \today\\
Students: & Axel Perschmann, Tarek Saier
\end{tabular}

\section*{Exercise 7.1}
\textbf{(a)} $\Pi'=\{V,I,O,\gamma\}$ with\\
\begin{itemize}
\item $V=\{above\h a,above\h b,above\h c,below\h a,below\h b,below\h c\}$\\
$\mathscr{D}_{above\h\Upsilon}=\{A,B,C,n\}\setminus\{\Upsilon\}$\\
$\mathscr{D}_{below\h\Upsilon}=\{A,B,C,t\}\setminus\{\Upsilon\}$\\
where $\Upsilon\in\{A,B,C\}$

\item $I(a)=1$ for $a\in \{below\h b=t,above\h b=A,above\h a=n,\\
\hphantom{wwww}below\h c=t,above\h c=n\}$\\
$I(a)=0$ else

\item $O=\{move\h X\h Y\h Z,move\h X\h Table\h Z,move\h X\h Y\h Table\}$\\
$move\h X\h Y\h Z=\langle (below\h X=Y) \land (above\h X=n) \land (above\h Z=n),$\\
\hphantom{wwww}$(above\h Y:=n) \land (below\h X:=Z) \rangle$\\
$move\h X\h Table\h Z=\langle(below\h X=t) \land (above\h X=n) \land (above\h Z=n),$\\
\hphantom{wwww}$(below\h X:=Z)\rangle$\\
$move\h X\h Y\h Table=\langle(below\h X=Y) \land (above\h X=n),$\\
\hphantom{wwww}$(above\h Y:=n)\ \land (below\h X:=t)\rangle$\\
for pair-wise distinct $X,Y,Z\in\{A,B,C\}$

\item $\gamma=(above\h c=B)\land(above\h a=C)$
\end{itemize}
And the addition\footnote{to make this a bit less verbose and better readable} that every $above|below[:]$$=\Upsilon$ with $\Upsilon\in\{A,B,C\}$ implies its counterpart (e.g. $above\h A[:]$$=B$ also tests/sets $below\h B[:]$$=A$).\\
\\
\textbf{(b)} $\Pi''=\Pi$\\
\\
\textbf{(c)}
\newpage

\section*{Exercise 7.2}
\textbf{(a)} Since both $h_1$ and $h_2$ include the blank tile, $h_1+h_2$ is not admissible.\\
Proof by counterexample:\\
$n=1$, $m=2$\\
\begin{tabular}{lll}
$s = \begin{matrix}
t_0 & t_1 & t_2 & t_3 \\
t_4 & t_5 & t_6 & t_7 \\
t_8 & t_9 & t_{10} & t_{11} \\
t_{12} & t_{13} & b & t_{14} \\
\end{matrix}$ &
$\alpha_2\text{-view}: \begin{matrix}
t_0 & - & - & - \\
- & - & - & - \\
- & - & - & - \\
- & - & b & - \\
\end{matrix}$ &
$\alpha_1\text{-view}: \begin{matrix}
- & t_1 & t_2 & t_3 \\
t_4 & t_5 & t_6 & t_7 \\
t_8 & t_9 & t_{10} & t_{11} \\
t_{12} & t_{13} & b & t_{14} \\
\end{matrix}$
\end{tabular}\\
\\
$h^*(s)=1$\\
$h_1(s)+h_2(s)=1+1=2$\\
$2>1$, therefore $h_1+h_2$ is not admissible.\\
\\
\textbf{(b)} Basic approach: since for all tiles $t_i$ it holds that $t_i\in h_3$ iff $t_i\not\in h_4$ and vice verca \emph{and} both $h_3$ and $h_4$ do \emph{not} include the blank tile, $h_3+h_4$ is admissible.\\
\\

\label{lastpage}
\end{document}
